\documentclass[a4paper, 10pt]{article}

%%Packages
\usepackage[utf8]{inputenc} %Usage des caractères speciaux
\usepackage[T1]{fontenc} %
\usepackage{enumerate}
\usepackage{natbib}
\usepackage[french]{babel}
\usepackage{amsmath,amsfonts}
\usepackage{graphicx}
\usepackage{blindtext}
	\blindmathtrue
\usepackage{geometry}
	\geometry
	{top    =2cm,
	bottom  =2.5cm,
	left    =2cm,
	right   =2cm}
\usepackage{multicol}
 %Intérêt de mettre une marge asymétrique lors de l'impression de fichiers
\usepackage{hyperref} % crée des liens hypertexte pour tout ce qui peut servir
\usepackage{fancyhdr} % cree un layout de page styley
	\pagestyle{fancy}
	\fancyhead[LE,RO]{Formation \LaTeX}
	%\fancyhead[RE,LO]{\thesection}
	\fancyfoot[CE,CO]{\leftmark}
	\fancyfoot[LE,RO]{\thepage}



%% Page de garde
\title{\LaTeX  Training}
\author{Leo}





%% Début du code
\begin{document}
\maketitle
\tableofcontents
\newpage
\begin{multicols}{2}

\section{1stPart}
but does it work?

\subsection{1.1}
éçà\& % le caractere & est un caractere de compilation
\subsubsection{1.1.1}


Ceci est un paragraphe long. 
Si je reviens à la  ligne seulement une fois Latex compile tout comme étant un seul paragraphe

En revanche si je laisse une ligne blanche entre mes deux paragraphe LateX lance un second paragraphe
\newline

La commande newline crée un espace à la suite du paragraphe précédent

Les grenouilles sont vertes !


\begin{itemize}
\item 1
\newline
\item 2 
\end{itemize}


\begin{enumerate}
\item ahaha
\item ahaha
\end{enumerate}


\section{les mathematiques avec \LaTeX}

Il existe deux grands types d'environnements de maths :
\begin{itemize}
\item 1
des maths en ligne:  il faut utilise le symbole dollar
exemple la fonction $f(\alpha) \geq 5$
\item 2
Des maths en standalone
$$
f(\alpha)=\zeta_i^j
$$
$$
{\alpha \beta }^6
$$
$$
\sum_{i=0}^\infty \zeta_{ik}^5
\mbox{hahahaha} % Permet d'écrire du texte dans les équations
$$

% brackets pour faire des anotations sous des termes
$$
f(\alpha)=\underbrace{A}_{term1} +\underbrace{B}_{term2}
$$

%fractions
$$
\frac{3}{2} \times \frac{16}{156} 
$$

% label equations
\begin{equation}
\upsilon=22
\label{eq:fractions}
\end{equation}
$\Upsilon$
% des caligraphies special math
$$
\mathcal R % typiquement pour les repères en math, ou les espaces
%\mathbb R
$$

%matrices
\begin{equation}
M=
\begin{pmatrix}

a & b \\
c & d
\end{pmatrix}
\end{equation}

% Afficher proprement les parenthèses
$$
a=
\left(
\dfrac{\alpha + \dfrac{beta}{\kappa}}{5+upsilon}
\right)
$$

% afficher des systèmes - avec la fonction split
$$
f(X)=
\left\lbrace
\begin{split}
5, \forall x \in \mathbb R^*\\
12, \forall x \in \lbrace 0 \rbrace
\end{split}
\right. % le point dit que il n'y a rien comme délimiteur à droite
$$

\begin{eqnarray}
a & =    & b + c\\
  & \neq & d + z
  % les caractère & permettent l'alignement ligne à ligne
\end{eqnarray}
%another test



\end{itemize}

% ON peut aussi donner des noms aux équations 



\section{commandes}
\newcommand{\myvec}[1]{\vec #1} % permet de créer ses propres fonctions pour mieux chosir comment écrire
par exemple on veut afficher des vecteurs comme $\myvec V$


\section{figures et graphiques}

\includegraphics[scale=0.5]{figures/hello.jpg}

On peut inclure des images même au sein d'un texte \includegraphics[scale=0.1]{figures/hello.jpg}, ou d'un tableau  comme ici


\begin{figure}
\begin{center}
\includegraphics[width=\columnwidth]{figures/hello.jpg}
\caption{HELLOW} % Titre de la figure
\end{center}
\end{figure}
On peut chercher à faire correspondre la taille de l'image à la page

% Pour mettre la figure au milieu de la page quand il y a deux colonnes
\begin{figure*}
\begin{center}
\includegraphics[width=.4\textwidth]{figures/hello.jpg}
\end{center}
\end{figure*}

\blindtext[10]


\end{multicols}
\newpage

\section{Tables}
\subsection{Objet tabulaire non flottant}
\begin{table}
\begin{tabular}{|c|c|c|}
\hline
truc& machin & ceci\\
\hline
\includegraphics[width=0.3\columnwidth]{figures/hello.jpg} & 0.1 & 0.2 \\
\hline
\end{tabular}
\caption{table non flottante}
\end{table}















\section{Bibliographie}
je cite cette citation \cite{PMID12643357} % cite 
test \citep{PMID12643357} % Cite entre parenthese
second test  \citet{PMID12643357}
\bibliographystyle{apalike}

\bibliography{sources}




\end{document}
